\section{Introduction}
\label{sec:intro}

\paragraph{The Challenge of Granular Predictive Maintenance.}
Introduce the context of PdM and its economic/safety significance. Discuss the shortcomings of existing binary or simple RUL prediction models, highlighting the need to understand intermediate degradation states in complex machinery like aircraft engines. Mention the CMAPSS dataset as a standard benchmark.

\paragraph{Our Hybrid Approach.}
Briefly introduce the core ideas: deriving data-driven degradation stages via clustering, using these stages for both classification and regression tasks, and combining the outputs into a practical Risk Score. Emphasize the novelty of generating custom labels and the hybrid nature of the prediction.

\paragraph{Contributions.}
Clearly list the main contributions, aligning with the points above. Use a compact list environment if desired:
\begin{itemize}
	\item A data-driven methodology using unsupervised clustering (KMeans, Agglomerative) to define multi-stage degradation labels for the CMAPSS dataset, independent of standard RUL.
	\item A hybrid predictive framework integrating classification for current health stage assessment and regression for time-to-transition/failure prediction.
	\item The development and evaluation of a Risk Score metric combining probabilistic failure prediction and temporal estimates for enhanced maintenance decision support.
	\item Extensive experimental validation on the CMAPSS dataset demonstrating the feasibility and potential benefits of the proposed approach.
\end{itemize}

\paragraph{Paper Organization.}
Section \ref{sec:related_work} reviews relevant literature on predictive maintenance, the CMAPSS dataset, clustering for health monitoring, and hybrid modeling approaches. Section \ref{sec:methodology} details our proposed four-phase methodology, covering data preprocessing, feature engineering, clustering for stage definition, classification modeling, regression modeling, and risk score formulation. Section \ref{sec:experiments} describes the experimental setup, including dataset specifics, implementation details, and evaluation metrics. Section \ref{sec:results} presents and discusses the results obtained for each phase across the different CMAPSS subsets. Finally, Section \ref{sec:conclusion} concludes the paper, summarizing the findings and suggesting directions for future work.
