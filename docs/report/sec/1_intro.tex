\section{Introduction}
\label{sec:intro}

% Motivation: Importance of Predictive Maintenance (PdM) in industries (aerospace, energy, manufacturing).
% Problem: Limitations of traditional PdM.
%   - Often binary (failure/no failure), missing gradual degradation.
%   - Reliance on predefined labels (like standard RUL) might not capture true operational degradation stages.
%   - Need for more granular insights for optimal maintenance planning.
\paragraph{The Challenge of Granular Predictive Maintenance.}
Introduce the context of PdM and its economic/safety significance. Discuss the shortcomings of existing binary or simple RUL prediction models, highlighting the need to understand intermediate degradation states in complex machinery like aircraft engines. Mention the CMAPSS dataset as a standard benchmark.

% Proposed Solution: Overview of the hybrid approach.
%   - Multi-stage failure labeling using clustering on raw sensor data (innovation).
%   - Combining classification (current stage) and regression (time to next stage/failure).
%   - Introduction of a Risk Score for decision support.
\paragraph{Our Hybrid Approach.}
Briefly introduce the core ideas: deriving data-driven degradation stages via clustering, using these stages for both classification and regression tasks, and combining the outputs into a practical Risk Score. Emphasize the novelty of generating custom labels and the hybrid nature of the prediction.

% Contributions: List the key contributions.
%   - Novel method for generating multi-stage degradation labels from CMAPSS sensor data using clustering.
%   - Hybrid framework combining classification and regression for stage and time prediction.
%   - Formulation and application of a Risk Score for maintenance decision-making.
%   - Comprehensive evaluation on the CMAPSS dataset.
\paragraph{Contributions.}
Clearly list the main contributions, aligning with the points above. Use a compact list environment if desired:
\begin{itemize}
	\item A data-driven methodology using unsupervised clustering (KMeans, Agglomerative) to define multi-stage degradation labels for the CMAPSS dataset, independent of standard RUL.
	\item A hybrid predictive framework integrating classification for current health stage assessment and regression for time-to-transition/failure prediction.
	\item The development and evaluation of a Risk Score metric combining probabilistic failure prediction and temporal estimates for enhanced maintenance decision support.
	\item Extensive experimental validation on the CMAPSS dataset demonstrating the feasibility and potential benefits of the proposed approach.
\end{itemize}

% Paper Structure: Outline the rest of the paper.
\paragraph{Paper Organization.}
Section \cref{sec:related_work} reviews relevant literature. Section \cref{sec:methodology} details our proposed methodology. Section \cref{sec:experiments} describes the experimental setup. Section \cref{sec:results} presents and discusses the results. Section \cref{sec:conclusion} concludes the paper.

% --- Optional: Figure Placeholder ---
% \begin{figure}[t]
%   \centering
%   \includegraphics[width=0.8\linewidth]{figures/concept_overview.pdf} % Replace with your concept figure
%   \caption{Conceptual overview of the proposed hybrid predictive maintenance framework.}
%   \label{fig:concept}
% \end{figure}
% ----------------------------------

% ======================================================================
