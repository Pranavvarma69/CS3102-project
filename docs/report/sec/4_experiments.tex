\section{Experiments}
\label{sec:experiments}
\subsection{Setup}
We evaluated our framework on the NASA CMAPSS turbofan engine degradation dataset \cite{cmapss_data}, specifically subsets FD001-FD004 representing different operational conditions and fault modes. To maintain temporal integrity, we utilized 5-fold GroupKFold cross-validation based on engine ID, ensuring all data points from a single engine remained entirely within either the training or validation set. Implementation used Python with libraries including scikit-learn, pandas, imbalanced-learn, Matplotlib, Seaborn, and XGBoost.

\subsection{Data Exploration}
Exploratory analysis revealed key insights: (1) sensor readings follow discernible patterns as engines approach failure; (2) several sensors (e.g., s1, s5, s6, s10, s16, s18, s19 for FD001) showed constant values and were removed; (3) remaining sensors displayed varying degradation patterns ideal for stage identification; and (4) significant class imbalance exists with later degradation stages being underrepresented. These findings guided our feature engineering and modeling approaches.

\subsection{Implementation Summary}
Following our methodology, we implemented the four-phase hybrid framework:

\begin{itemize}
    \item \textbf{Phase 1 (Clustering):} We tested K-Means and Agglomerative clustering algorithms, selecting K-Means ($k=5$) due to its lower hyperparameter tuning requirements and training efficiency. Clusters were mapped to degradation stages (0-4) based on RUL averages and validated through visual inspection of sensor trends and dimensionality reduction visualizations.

    \item \textbf{Phase 2 (Classification):} We trained KNN, Gaussian Naive Bayes, and Logistic Regression models to predict the `kmeans\_stage`. These algorithms were chosen for their interpretability and relatively few hyperparameters, enabling faster training. We addressed class imbalance using SMOTE within cross-validation folds and/or `class\_weight='balanced'`. Hyperparameters were tuned via RandomizedSearchCV or GridSearchCV, optimizing for macro F1-score.

    \item \textbf{Phase 3 (Regression):} Ridge Regression and XGBoost models were trained to predict `time\_to\_next\_stage`. Features included rolling statistics, lag features, selected raw sensors, and the predicted `kmeans\_stage`. Hyperparameters were tuned optimizing for negative RMSE.

    \item \textbf{Phase 4 (Risk Score):} Calculated using the probability of Stage 4 and the predicted stage with our risk formula. Scores were Min-Max normalized, with alerts triggered above a threshold determined through validation performance.
\end{itemize}

\subsection{Evaluation Metrics}
Performance was assessed through multiple lenses: (1) for clustering, we used visual inspection of PCA/t-SNE projections, sensor trend analysis, and stage distributions; (2) for classification, we measured macro/weighted F1-scores, per-class precision, recall, and F1, with special attention to performance on critical stages 3 and 4; (3) for regression, we calculated RMSE, MAE, and R², complemented by actual vs. predicted and residual plots; and (4) for the risk score, we examined temporal risk trends and alert precision/recall metrics.
